% --------------------------------------------------------------
% This is all preamble stuff that you don't have to worry about.
% Head down to where it says "Start here"
% --------------------------------------------------------------
 
\documentclass[12pt]{article}
 
\usepackage[margin=1in]{geometry} 
\usepackage{amsmath,amsthm,amssymb,amsfonts, multirow, graphicx, float}
 
\newcommand{\N}{\mathbb{N}}
\newcommand{\Z}{\mathbb{Z}}
 
\newenvironment{Problem}[2][Problem]{\begin{trivlist}
\item[\hskip \labelsep {\bfseries #1}\hskip \labelsep {\bfseries #2.}]}{\end{trivlist}}

\newenvironment{Solution}[2][Solution]{\begin{trivlist}
\item[\hskip \labelsep {\bfseries #1}\hskip \labelsep {\bfseries #2}]}{\end{trivlist}}

 
\begin{document}
 
% --------------------------------------------------------------
%                         Start here
% --------------------------------------------------------------
 
%\renewcommand{\qedsymbol}{\filledbox}
 
\title{Chapter 4/5 Homework 3}%replace X with the appropriate number
\author{Lucas Gallegos\\ %replace with your name
ME 397 - Introduction to Robot Modeling and Control} %Course title
\maketitle
 
\begin{Problem}{1} 

Derive the Jacobian for your previous RRP robot symbolically.

\begin{Solution}{}
\end{Solution}



$J = $\left(\begin{array}{ccc} -d_{3}\,\cos\left(\theta _{2}\right)\,\sin\left(\theta _{1}\right) & -d_{3}\,\cos\left(\theta _{1}\right)\,\sin\left(\theta _{2}\right) & \cos\left(\theta _{1}\right)\,\cos\left(\theta _{2}\right)\\ d_{3}\,\cos\left(\theta _{1}\right)\,\cos\left(\theta _{2}\right) & -d_{3}\,\sin\left(\theta _{1}\right)\,\sin\left(\theta _{2}\right) & \cos\left(\theta _{2}\right)\,\sin\left(\theta _{1}\right)\\ 0 & d_{3}\,\cos\left(\theta _{2}\right) & \sin\left(\theta _{2}\right)\\ 0 & \sin\left(\theta _{1}\right) & 0\\ 0 & -\cos\left(\theta _{1}\right) & 0\\ 1 & 0 & 0 \end{array}\right)

\begin{itemize}
\item The DH parameters for this problem were found in the previous homework assignment. Using these DH Parameters, along with equations in the textbook, I solved the forward kinematics symbolically to compute the Jacobian Matrix symbolically. This can be found in the MATLAB script titled "hw3 rrp symbolic.m".
\end{itemize}

\end{Problem}

\pagebreak

\begin{Problem}{2}
Derive the Jacobian for a planar RRR manipulator.

\begin{Solution}{}
\end{Solution}

\footnotesize
\setlength{\arraycolsep}{2.5pt}
\medmuskip = 1mu % default: 4mu plus 2mu minus 4mu


$J=$\left(\begin{array}{ccc} -L_{2}\,\sin\left(\theta _{1}+\theta _{2}\right)-L_{1}\,\sin\left(\theta _{1}\right)-L_{3}\,\sin\left(\theta _{1}+\theta _{2}+\theta _{3}\right) & -L_{2}\,\sin\left(\theta _{1}+\theta _{2}\right)-L_{3}\,\sin\left(\theta _{1}+\theta _{2}+\theta _{3}\right) & -L_{3}\,\sin\left(\theta _{1}+\theta _{2}+\theta _{3}\right)\\ L_{2}\,\cos\left(\theta _{1}+\theta _{2}\right)+L_{1}\,\cos\left(\theta _{1}\right)+L_{3}\,\cos\left(\theta _{1}+\theta _{2}+\theta _{3}\right) & L_{2}\,\cos\left(\theta _{1}+\theta _{2}\right)+L_{3}\,\cos\left(\theta _{1}+\theta _{2}+\theta _{3}\right) & L_{3}\,\cos\left(\theta _{1}+\theta _{2}+\theta _{3}\right)\\ 0 & 0 & 0\\ 0 & 0 & 0\\ 0 & 0 & 0\\ 1 & 1 & 1 \end{array}\right)

\begin{itemize}
\item Using the DH Parameters for a planar RRR robot shown in Table 1, along with equations in the textbook, I solved the forward kinematics symbolically to compute the Jacobian Matrix symbolically. This can be found in the MATLAB script titled "hw3 rrr symbolic.m".
\end{itemize}

\begin{table}
\caption{DH Parameters for planar RRR Robot}
\begin{center}
\begin{tabular}{ c| c c c c }
 & $\theta$ & d & a & $\alpha$ \\ 
 \hline
 $A_1$ & $\theta_1$\textsuperscript{*} & 0 & $a_1$ & 0 \\  
 $A_2$ & $\theta_2$\textsuperscript{*} & 0 & $a_2$ & 0 \\
 $A_3$ & $\theta_3$\textsuperscript{*} & 0 & $a_3$ & 0
\end{tabular}
\end{center}
\end{table}

\end{Problem}

\pagebreak

\begin{Problem}{3}
Animate a planar RRR robot and the RRP and 7DOF robots from the previous homework assignment using any trajectory.

\begin{Solution}{}
\end{Solution}

\begin{itemize}
\item I employed a quintic polynomial trajectory. See the attached .avi videos.
\end{itemize}

\end{Problem}



 
% --------------------------------------------------------------
%     You don't have to mess with anything below this line.
% --------------------------------------------------------------
 
\end{document}